%%===================================================================================
%%
%%    Description:  Functional specification for the market_data_grabber program.
%%
%%        Version:  0.1
%%        Created:  09/19/2016
%%       Revision:  none
%%
%%         Author:  Herbert Nowell (herb@herbertnowell.com) 
%%   Organization:  
%%      Copyright:  
%%
%%          Notes:  
%%                
%%===================================================================================

\documentclass{report}

\usepackage[pass,letterpaper]{geometry}

\begin{document}
\title{Market Data Grabber Functional Specification}
\author{Herbert H. Nowell}

\maketitle

\section{About this Specification}

This specification is a starting point for the development of Market Data
Grabber 0.1 which will focus only on equity prices, splits and dividends, and
equity index values.  As time goes on aspects will change.  At the outset I
suspect the back end database will change many times for example.  While
efforts will be made to keep the spec up to date at times code may lead the
document.

\section{Overview}

Market data grabber is a component of the portfolio manager software suite.
The name is open to change to something more manageable even if not more
sonorous.

\section{Major Features}

\begin{itemize}
    \item Download stock data including closes, highs and lows by days, splits, 
          and dividends.  
    \item It will be able to grab multiple stocks in one invocation and for 
          multiple days.  
    \item Will be able to download stocks on major indices as a whole: Russell, 
          S\&P, and Dow.
    \item Be driven by command line or configuration file.
\end{itemize}

\section{Design Goals}

The biggest goal is a system that can be 100\% automated.  The principle purpose is
overnight runs after market closes to support analysis runs.

\section{Components}

\begin{itemize}
    \item Perl script - does all setup and retrieval
    \item Database - Some form of SQL accessable DB including scripts to create
          the DB.
\end{itemize}

\section{User Interface}

The program will be a command line program, \texttt{market\_data\_grabber} written in
Perl.  As its default it will take as its arguments a list of stock tickers.  If the
ticker is for an index known to the program it will download all stocks associated
with that index for the dates given.

It will have the following command line switches:
\begin{itemize}
    \item {\texttt -h/--help} and {\texttt -v}: The former will also be automatically
        invoked if the arguments given do not match those expected.

    \item {\texttt --begin-date} and {\texttt --end-date}: If neither is used only 
        the most recent business day prior to today is captured.  If the former only
        is used the ranges begins on that day and ends on the most recent business
        day.  The latter can only be used in conjunction with the former and sets 
        the ending business day to be captured.

    \item {\texttt --cfg-file} option that specifies the argument is a file listing 
        what to download.  This will be a yaml file whose structure will be such that
        it can include information on what to download about that stock, more or less
        than the default, and set date ranges by ticker.
\end{itemize}

\section{Data Sources}

The initial target data sources are the ones in the Quant Stack Exchange's master 
list of data sources \texttt{https://quant.stackexchange.com/questions/141/}
in the equities section that are available free.

\section{Licenses}

This document is provided for reference only and is not licensed.  The program will
be licensed under Creative Commons Attribution 4.0 International License 
(\texttt{(CC BY 4.0)}).  See \texttt{https://creativecommons.org/licenses/by/4.0/}.

\end{document}




